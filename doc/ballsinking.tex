\documentclass[12pt,halfline,a4paper]{ouparticle}

\begin{document}

\title{Ball Sinking into a Viscous Fluid} 

\author{%
\name{Li Ruixiang}
\email{\texttt{liruixiang@zju.edu.cn}}
}
\date{}

\maketitle

\section{Motion of a Ball Sinking into a Viscous Fluid}
一个刚体在液体中的运动由下列的运动方程描述:
\begin{equation}\begin{aligned}
    & \rho_{p}V_{c}\dot{\mathbf{u}}_{c}=\rho_{f}\oint_{\partial S}\boldsymbol{\tau}\cdot\mathbf{n}\mathrm{d}\sigma+(\rho_{p}-\rho_{f})V_{c}\mathbf{g}, \\
    & I_{c}\dot{\boldsymbol{\omega}}_{c}=\rho_{f}\oint_{\partial S}\mathbf{r}\times(\boldsymbol{\tau}\cdot\mathbf{n})\mathrm{d}\sigma,
   \end{aligned}\end{equation}
其中$\rho_{p}$是刚体的密度,$V_{c}$是刚体的体积,$\mathbf{u}_{c}$是刚体的质心速度,$\rho_{f}$是流体的密度,$\boldsymbol{\tau}$是应力张量,$\mathbf{n}$是单位法向量,$\mathbf{g}$是重力加速度,$I_{c}$是刚体的转动惯量,$\boldsymbol{\omega}_{c}$是刚体的角速度,$\mathbf{r}$是刚体上的点到质心的矢量,$S$是刚体的表面.
对于不可压流体,应力张量$\boldsymbol{\tau}$可以表示为
$$\boldsymbol \tau = -pI + \nu (\nabla \boldsymbol{u} + (\nabla \boldsymbol{u})^T)$$

\section{Coupling of the Solid and the Fluid}
固液边界的耦合条件为无滑移边界条件,即液体和固体之间的相对速度为0.
固体的速度可以表示为平动和转动的合成:
\begin{equation}\mathbf u(x)=\mathbf{u}_c+\boldsymbol{\omega}_c\times(x-x_c),\end{equation}




\end{document}